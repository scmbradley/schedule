\documentclass{conference}



\SetTitle{Quantum Gravity in Perspective}
\SetInstitute{Munich Centre for Mathematical Philosophy}
\SetDates{31 May -- 1 June}
\SetWebsite{http://www.mcmp.philosophie.uni-muenchen.de/index.html}
\SetVenue{Room M210, 2nd floor, LMU main building, Geschwister-Scholl-Platz 1, D-80539 Munich}

\begin{document}

\MakeTitle


\begin{Day}{31 May}
  \AddTalk{kiefer}{Claus Kiefer (Cologne)}
    {Quantum Geometrodynamics -- Whence, Whither}{09.00--09.50}
        {Quantum geometrodynamics is canonical quantum gravity with the
three-metric as the configuration variable. Its central equation is
the Wheeler--DeWitt equation. Here I give an overview of
the status of this approach. The issues discussed
include the
problem of time, the relation to the covariant theory, the
semiclassical approximation as well as applications to black holes and
cosmology. I conclude that quantum
geometrodynamics is still a viable approach and provides insights into
both the conceptual and technical aspects of quantum gravity.}

  \AddTalk{erdmenger}{Johanna Erdmenger (Heisenberg Institute)}
    {New dualities between gauge theories and gravity}{09.50--10.40}
    {Based on string theory, the AdS/CFT correspondence (AdS: Anti-de Sitter space, CFT: Conformal field theory) provides a new map between quantum gauge theories and classical gravity theories. This new map arises from a specific low-energy limit of string theory. It is a duality, mapping strongly coupled quantum gauge theories to weakly coupled gravity theories. Moreover, this duality maps classical gravity theories on five-dimensional Anti-de Sitter space to quantum gauge theories on the four-dimensional boundary of this space, a property referred to as holography. In addition to its significance for quantum gravity, the AdS/CFT correspondence has been generalized to further examples of gauge/gravity duality. These have practical applications to the study of systems relevant in elementary particle and condensed matter physics. We will give an introduction to gauge/gravity duality.}

  \AddBreak{10.40--11.00}{Coffee Break}

  \AddTalk{hossenfelder}
    {Sabine Hossenfelder (NORDITA)}
    {Experimental Search for Quantum Gravity}{11.00--12.00}
    {The development of a theory for quantum gravity cannot proceed
without phenomenology. In this talk I will lay out the methodology
of the phenomenology of quantum gravity, and the difficulties
this research area faces due to the peculiarity of the subject. I
will then discuss some examples of models for quantum gravity 
phenomenology and possible experiments to test them.}

  \AddTalk{helling}{Robert Helling (LMU Munich)}
  {The ``shut up and calculate'' approach to quantum gravity}
  {12.00--12.50}
  {Quantum mechanics would have never been developed requiring to first understand all foundational issues. Rather, it was approached in a pragmatic way summarized as “shut up and calculate”. For quantum gravity we already understand a number of its properties even if it is still unclear what form the final theory will have and how it solves its conceptual conundrums.

The often repeated claim that quantum gravity lacks empirical data is --- taken without qualification --- not true. Rather, the requirement to reduce to known theories relativity in the appropriate limits together with our experience of everyday physics rule out most “creative” proposals.  Even tiny modifications at the Planck scale (like breaking of Lorentz invariance, non-commutativity or massive gravity) are communicated via quantum effects to all scales ruining agreement with the world at much lower energy scales. In effect, we do not suffer too many but rather too few proposals for a theory of quantum gravity.

We want to pursue the effective-field-theory line-of-thought neglecting conceptual expectations. The effective field theory approach still leaves room for a wide variety of ideas like string theory, loop quantum gravity or emergent gravity as we will show in a number of examples.}

  \AddBreak{12.50--14.00}{Lunch Break}

  \AddTalk{lehmkuhl}{Dennis Lehmkuhl (Wuppertal)}
    {Einstein's Approach to Quantum Gravity}{14.00--14.50}
    {It is common knowledge that despite being a pioneer of the early quantum theory, Einstein opposed the probabilistic interpretation of the new quantum mechanics of 1925/1926, and he would have opposed any approach of a “quantization of gravity” relying on this interpretation. What is less well-known is that Einstein had an alternative approach that bears some similarities to more recent ideas to “general relativize quantum mechanics” rather than “quantizing general relativity”. Einstein identified discreteness in nature with quantum theory: the existence of photons, the quantisation of electric charge, etc. Roughly speaking, his idea was that these quantum features of reality could be derived from a generally covariant field theory of gravitation and electromagnetism by finding overdetermined partial differential equations which allow for solutions capable of representing quantum particles. If this were to be achieved, the “groundfloor ontology” of the world would have turned out to be one of classical fields described by generally covariant partial differential field equations, while “the quantum” would only enter on the level of the solutions to these equations.

The talk reviews the above research programme of Einstein’s, and reflects on whether it can lend some inspiration to modern quantum gravity research.}

  \AddTalk{blum}{Alexander Blum (MPIWG)}{The moving frontier: How Quantum Gravity became the great unsolved problem of modern physics}
    {14.50--15.50}
    {The immense difficulties encountered in finding a quantum theory of gravity are generally understood as revealing a deep conceptual divide between the two major physical theories of the 20th century: quantum mechanics (or quantum field theory) and the general theory of relativity.
It was, however, far from clear from the outset (the outset here being physics after the quantum and relativity revolutions) that this would be the essential fault line of modern physics. It seemed rather, all through the 1930s and 1940s that there was a conceptual incompatibility between quantum theory and field theories in general (which included general relativity as a rather esoteric and empirically irrelevant example), revealed through the divergence difficulties of the attempted quantum field theories.

In my talk, I will attempt to trace how the perceived conceptual divide shifted in the course of the Twentieth Century. It is hoped that this analysis can simultaneously address general historical questions of scientific development, especially at the boundaries of distinct well-established theories, and at the same time help understand what the unique features of quantum gravity are that have established it, in the course of the last century, as the prime unsolved problem of theoretical physics.}

  \AddBreak{15.50--16.10}{Coffee Break}

  \AddTalk{khavkine}{Igor Khavkine (Radboud)}{Gravity: An Exercise in Quantisation}{16.10--17.00}
    {The quantization of General Relativity (GR) is an old and challenging prob- lem that is in many ways still awaiting a satisfactory solution. GR is a partic- ularly complicated field theory in several respects: non-linearity, gauge invari- ance, dynamical causal structure, renormalization, singularities, infrared effects. Fortunately, much progress has been made on each of these fronts. Our under- standing of these problems has evolved greatly over the past century, together with our understanding of quantum field theory (QFT) in general. Today, the state of the art in QFT knows how to address each of these challenges, as they occur in isolation in other field theories. There is still an active research program aiming to combine the relevant methods and apply them to GR. But, at the very least, the problem of the quantization of GR can be formulated as a well defined mathematical question. On the other hand, quantum GR also faces a different set of obstacles: “timelessness,” “non-renormalizability,” “naturality,” “unification,” which reflect, not its technical difficulty, but rather the aesthetic and philosophical preferences of practicing theoretical physicists.

I will briefly discuss how the technical state of the art and a scientifically conservative philosophical position make these obstacles irrelevant. Time per- mitting, I will also briefly touch on some aspects of the state of technical state of the art that have turned the quantization of GR into a (still challenging) exercise: covariant Poisson structure, BV-BRST treatment of gauge theories, deformation quantization, Epstein-Glaser renormalization.}

  \AddTalk{teh}{Nicholas Teh (Cambridge)}{Philosophical Perspectives on QG from Lower Dimensions}{17.00--17.50}
  {In this talk, I will attempt to bring the philosophical literature on QG (much of which turns on the philosophical interpretation of gauge symmetry) to bear on the exploration of the subject in (2+1) dimensions. I will emphasize algebraic and category-theoretic structures.}

  \AddBreak{19.00--21.00}{Workshop Dinner at Kaisergarten}
\end{Day}

\begin{Day}{1 June}
  \AddTalk{wuthrich}{Chris Wuthrich (UCSD)}{Time and space in causal set theory}{09.00--09.50}
{Causal set theory offers an elegant and philosophically rich, though admittedly inchoate, approach to quantum gravity. After presenting its basic theoretical framework, I will show how space and time vanish from the fundamental picture it offers. The absence of space and time from the theory raises the serious question of whether such a theory can be empirically coherent at all, i.e., whether its truth would not undermine any justification we may have for believing it. If it can be shown that spacetime re-emerges from the fundamental structure in the appropriate limit, I will argue, then the threat of empirical incoherence is averted and it can be appreciated how space and time emerge from what there is, fundamentally, according to causal set theory. I shall close by sketching the prospects of the antecedent of this conditional claim. }

  \AddTalk{oriti}{Daniele Oriti (Einstein Institute)}{Dissappearance and emergence of space and time in quantum gravity}{09.50--10.50}
  {We recall the hints for the disappearance of continuum space and time at microscopic scales, coming from classical and semi-classical gravitational physics. These include arguments for discreteness or for a fundamental non-locality, in a quantum theory of gravity. We compare how these ideas are realized in specific quantum gravity approaches, and focus in particular on the group field theory formalism, itself strictly related to other approaches, in particular loop quantum gravity.

Next, we consider the emergence of continuum space and time from the collective behaviour of discrete, pre-geometric and non-spatio-temporal “atoms” of quantum space. After discussing the notion of “emergence”, with Bose condensates as one paradigmatic example, and some specific conceptual difficulties with the notion of “emergent spacetime”, we argue for spacetime as a kind of ”condensate”, the result of a phase transition, physically identified with the big bang.

We then illustrate recent results, in the context of the group field theory framework, establishing a tentative procedure for the emergence of cosmological (homogeneous) spacetimes and their effective quantum dynamics from fundamental, pre-geometric models.
Last, we re-examine the conceptual issues raised by the emergent spacetime scenario in light of this concrete example.}

  \AddBreak{10.50--11.10}{Coffee Break}

  \AddTalk{pitts}{J. Brian Pitts (Cambridge)}
    {Real Change Happens in Hamiltonian Relativity; Just Ask the Lagrangian (about Time-like Killing Vectors, First-Class Constraints and Observables)}
    {11.10--12.00}
    {In Hamiltonian GR, change has seemed absent.  Attention to the gauge generator G facilitates a neglected calculation:  a first-class constraint generates a bad physical change in electromagnetism and GR, spoiling the constraints, Gauss's law or the momentum and Hamiltonian constraints in the (physically relevant) velocities.  Only as a team G do first-class constraints generate a gauge transformation.  
To find change, insist on Hamiltonian-Lagrangian equivalence.  Change is ineliminable time dependence; in vacuum GR it is the absence of a time-like Killing vector field.  Neglecting spatial dependence, invariantly something depends on time via Hamilton's equations iff there is no time-like Killing vector. 

According to Bergmann, reality is not confined to observables, defined as both gauge invariant (hence real) and economical (Cauchy data on space).  Thus change can exist outside observables.

Bergmann’s lemma that observables have vanishing Poisson brackets for gauge transformations was imported by analogy to electromagnetism, neglecting the external vs. internal distinction and Hamiltonian-Lagrangian equivalence.  The resulting implausible Killing-type condition lacks the local examples required by Bergmann.  Taking observables to be geometric objects (tensors, etc.) as usual in the 4-dimensional Lagrangian formalism makes the Poisson bracket of G with an observable the Lie derivative of a geometric object (on-shell):  covariance, not invariance.}

  \AddTalk{thebault}{Karim Th{\'e}bault (LMU Munich)}{Time Remains}{12.00--12.50}
  {Even classically, it is not entirely clear how one should understand the implications of general covariance for the role of time in physical theory. On one popular view, the essential lesson is that change is relational in a strong sense, such that all that it is for a physical degree of freedom to change is for it to vary with regard to a second physical degree of freedom. This implies that there is no unique parameterization of time slices, and also that there is no unique temporal ordering of states.  At a quantum level this approach to general relativity is generally understood to lead to a universe eternally frozen in an energy eigenstate. Here we will start from a different interpretation of the classical theory, and in doing so show how one may avoid this acute `problem of time' in quantum gravity. Under our view, duration is still regarded as relative, but temporal succession is taken to be absolute. This is consistent with general covariance because it can be maintained only by the addition of an arbitrary time parameter corresponding to the minimal temporal structure necessary for a succession of observations to be represented. This approach to the classical theory of gravity is argued to then lead to a relational quantization methodology, such that it is possible to conceive of dynamical observables within a theory of quantum gravity.}

  \AddBreak{12.50--14.00}{Lunch Break}

  \AddTalk{bouatta}{Nazim Bouatta (Cambridge)}{On emergence and reduction in string theory}{14.00--14.50}
  {Emergence and reduction are compatible, despite the widespread "ideology" that they contradict each other. The overarching theme is that the reconciliation of emergence and reduction turns on subtle uses of infinite limits. Previous philosophical literature about how one theory or a sector, or regime, of a theory might be emergent from, and-or reduced to, another one has tended to emphasise cases, such as occur in statistical mechanics. But here, we will develop this viewpoint for quantum field theories, including those on the way to the 't Hooft limit. This aspect relates closely to the string-gravity connection.}

  \AddBreak{14.50--15.10}{Coffee Break}

  \AddTalk{dawid}{Richard Dawid (Vienna)}{String Theory, Final Theory Claim and Scientific Realism}{15.10--16.10}
    {T-duality, which is an important feature of string theory, implies that
the string scale constitutes a minimal length scale: within the conceptual
framework of the theory, no higher energy scale can provide new
information. This in turn may be taken to suggest that, if ST is valid at
its own characteristic scale, no new theories which become empirically
distinguishable from string theory at higher energy scales should be
expected to be found. The problem with this ‘final theory claim’ hinges on
the fact that it seems to beg the question by presupposing the conceptual
framework the finality of which it aims to establish. In the talk I aim at
understanding significance and limitations of string theory’s final theory
claim based on the distinction between local and global limitations to
scientific underdetermination. It will be argued that the final theory
claim does carry some weight but cannot be understood independently from
more general considerations about the scientific role of limitations to
underdetermination. Some implications of final theory claims for the
scientific realism debate are discussed in the second part of the talk.}

  \AddTalk{rickles}{Dean Rickles (Sydney)}{Dualities and the Physical Content of Theories}{16.10--17.10}
{A duality expresses a relationship between a pair of putatively distinct physical theories. Theories are said to be dualwhen they generate 'the same physics', where “same physics” is parsed in terms of, e.g., having the same amplitudes, expectation values, observable spectra, and so on. Hence, theories related by dualities can look very different while making exactly the same predictions about observable phenomena. Indeed, such theories can look sufficiently different that would-be interpreters would surely consider them to be representations of very different possible worlds. In this talk I will be concerned with the question of whether dualities reflect some deep aspect of reality, or whether they are simply a formal device that aids computations in difficult contexts (functioning in much the same way as a change of variables).  This links quite naturally to problems of underdetermination, and also to the issue of what we mean by theory in such contexts. I defend a rather deflationary account of the philosophical implications of dualities. Since the underdetermination is taken to apply to structurally distinct theories (that is, it is structure that is underdetermined), I will also consider whether dualities pose a problem for structural realists.}
\end{Day}

\pagebreak
\ListAbstracts

\end{document}
